\section{Threats to validity}\label{sec:threats}

Our type system and implementation suffer some unsoundness.  The most
important is that they do not capture all sources of nondeterminism.  One
important such source is concurrency.
%
In addition, our annotations of the JDK (the Java standard library) are incomplete.
Some of the unannotated parts might introduce nondeterminism without
the verification tool being aware of it.

The type system presented in this paper does not account for the fact that
a nondeterministic conditional expression (in an \<if> or \<switch>
statement or a \<for> or \<while> loop, for example) may behave differently
from execution to execution.
These ``implicit flows'' are a well-known challenge for taint analysis.
Suppose that an expression used in a conditional is tainted.
A standard approach
\cite{Kang2011DTADT,Arzt:2014:FPC:2594291.2594299}
%\todo{Add citations}
 for a dynamic taint analysis is to taint any value that may
be assigned in either branch of the conditional statement.  This tends to rapidly lead to most of
the program state being tainted.
In a static analysis, every possibly-assigned lvalue would be considered
possibly tainted; again, this tends to snowball through the program.

Our implementation, \theDeterminismChecker, can enforce that conditional
expressions are deterministic, but we found this to be too restrictive:  it
causes \theDeterminismChecker to issue large numbers of false positive
warnings, or it forces types to be \<Det> unnecessarily.
We disabled this check in most of our case study.

It our case study, we also disabled a check that required all caught
exceptions to have \<NonDet> type.  This type is a safe approximation to an
exception that can be thrown from arbitrary code, including libraries.
Enabling this check led to many unhelpful false positive warnings, which
are only relevant if unverified libraries throw nondeterministic exceptions.


\TheDeterminismChecker only examines the code it is run on.  Unchecked
libraries and native code with incorrect specifications might introduce
nondeterminism even if \theDeterminismChecker issues no warnings.


We have performed only one case study.  This case study found important
previously-unknown errors, but its results might not carry over to other
programs or to programs written in other programming lanugages.  We
mitigated this problem by showing that \theDeterminismChecker finds many
of the sources of nondeterminism identified by other tools.
