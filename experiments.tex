\section{Comparison to NonDex}

The state of the art in flaky test detection is NonDex~\cite{nondex}.
\Cref{sec:related} explains how NonDex works. This section compares the
errors reported by NonDex and \theDeterminismChecker.

\subsection{Case study with NonDex}\label{sec:nondex-randoop}

We ran NonDex on the same commit of Randoop as we used for the case study
of \cref{sec:randoop-case-study}.  This version contains all the nondeterminism bugs that
\theDeterminismChecker found.

We modified Randoop by deleting tests that were skipped by its buildfile,
because the NonDex Gradle plugin does not respect those settings.
After that, NonDex ran without problems on Randoop.
NonDex reported no flaky tests.
%\todo{Revise the following text.
%  I renumbered the bugs without realizing that this section refers to them
%  by number, and I don't know which one these refer to any
%  more.
%  Please use cross-references or macros.  Also, this needs to be expanded
%  to account for the additional bugs.}
We investigated the tests covering code that \theDeterminismChecker reported as nondeterministic.
\bugHashSet and \bugClasspath have corresponding test cases whereas \bugTimestampOutput and 
\bugTimestampOutput do not.
As NonDex is a \textit{flaky test} detection tool, detecting \bugTimestampOutput and \bugTimestampOutput 
is outside of its scope.
The reason for nondeterminism in \bugClasspath was a call to the \<System.getProperty()> method which
is not modeled by NonDex.


\subsection{\TheDeterminismChecker on NonDex benchmarks}\label{sec:nondex-benchmarks}

\begin{figure}
    \begin{tabular}{lll}
        \<Class> & : \<getDeclaredFields>, \<getDeclaredMethods>, \<getFields> \\
        \<DateFormatSymbols> & : \<getZoneStrings> \\
        \<HashMap> & : \<entrySet>, \<keySet>, \<values>
    \end{tabular}
\caption{Souces of flakiness known to NonDex~\cite{nondex}.}
\label{fig:flaky-sources}
\end{figure}

\Cref{sec:nondex-randoop} shows that \theDeterminismChecker finds errors that NonDex does not.
This section determines whether NonDex finds errors that \theDeterminismChecker does not.
Its authors ran NonDex on 195 open-source projects, and NonDex found flaky tests in
21 of them~\cite{nondex}.
The authors also reported the sources of flakiness
after manually inspecting these tests. 
The flakiness that NonDex found was due to 7 methods in 3 classes (\cref{fig:flaky-sources}).

We tried to run \theDeterminismChecker on all these benchmarks, at the
commit given in the NonDex paper.
Some of the projects had moved, or did not build because their dependencies
had moved or were no longer available.  We repaired all these issues and
were able to compile all but two
projects, handlebars and oryx.  Some of the remaining projects did not pass
their tests, but that did not hinder us since \theDeterminismChecker works
at compile time.

For three of the projects, we could not find the flakiness reported in the
NonDex paper. The reported root cause of flakiness in \<easy-batch> and
\<vraptor> was a call to \<Class.getDeclaredFields>. We couldn't find a call
to this method in any of the source files of these two repositories.
Similarly, the flakiness in \<visualee> was attributed to an invocation of
\<Class.getDeclaredMethods> which we didn't find in the source code.
%\todo{Important: give a sentence description of each one.}

This left 16 projects.  We ran \theDeterminismChecker on the part of the
project that the NonDex authors determined as flaky.  In every case,
\theDeterminismChecker issued a warning on the nondeterministic code.  In
other words, \theDeterminismChecker's recall was 100\%.

All these benchmarks and instructions to reproduce our results can be found at \myurl{https://github.com/t-rasmud/determinismCaseStudies}.


\begin{figure}
    \centering
    \begin{subfigure}[b]{0.95\textwidth}
        \begin{verbatim}
    static public FieldAccess get(Class type) {
        ...
        while (nextClass != Object.class) {
            Field[] declaredFields = nextClass.getDeclaredFields();
            ...
        }
    }
        \end{verbatim}
        \caption{Nondeterministic source code from reflectasm.\vspace{0.5cm}}
        \label{code-reflectasm}
    \end{subfigure}

    \begin{subfigure}[b]{0.95\textwidth}
        \begin{verbatim}
    protected SimpleDataEvent createNextEvent() {
        for (Entry<String, FieldType> entry : fields.entrySet()) {
            ...
        }
        ...
    }
        \end{verbatim}
        \caption{Annotated source code from ActionGenerator.}
        \label{code-actiongenerator}
    \end{subfigure}    
    \caption{Errors detected by \TheDeterminismChecker.}
    \label{fig:nondex-source}
\end{figure}


\Cref{fig:nondex-source} shows samples of nondeterministic code from these
benchmarks.
We annotated the benchmarks based on 
assumptions made downstream of the shown code.

In reflectasm (\cref{code-reflectasm}), we changed the type of field
\<declaredFields> from \<Field[]> to \<@Det Field @Det []>.  That type
means a deterministic array of deterministic \<Field>, analogously to
\codeid{@Det List<@Det Field>}.  Then, \theDeterminismChecker issued a
warning at the assignment, because \<getDeclaredMethods> returns \<@Det
Field @Det []>, which is an order-nondeterministic array of deterministic
\<Field>s.

The ActionGenerator code (\cref{code-actiongenerator}) is similar.  Other
code assumes that \<entry> is deterministic, but annotating it as \<@Det>
leads to a warning from \theDeterminismChecker that iterating over
\<fields.entrySet()> (which is itself \<@OrderNonDet>) yields \<@NonDet>
\<Entry> values..

Nondex found 14 flaky tests in Apache Commons Lang due to calls
to \<Class.getDeclaredFields>. 
There are only 5 invocations of \<Class.getDeclaredFields>, so annotating 5 lines of source code
would have been sufficient to identify all that nondeterminism.
Having said that, we admit that there could be
significant programmer effort involved in annotating the whole program. On
the other hand, the NonDex authors
state ``we found that manually inspecting these failures was
rather challenging, and we leave it as future work to automate
debugging test failures.''
\TheDeterminismChecker reports source locations
which makes it easier for the programmer to fix issues, and the annotation
effort serves as valuable documentation and prevents regressions.

\TheDeterminismChecker is capable of detecting nondetermism in test
cases, as some of these examples are.
We wrote specifications for the
JUnit library (in addition to the JDK and some other libraries).
The specifications are the \<@Det> type qualifier on the formal parameters
to the \<assert>* methods.
\TheDeterminismChecker reports any \<junit> test that passes nondeterministic arguments to
\<assert>*, even if the test case is not executed.

%\todo{Link to det checker repo?}
%\todo{If there is time, we could do a second case study, on a program that
%  NonDex did not find nondeterminism in.  Our hope would be to find
%  nondeterminism that NonDex did not.  Choose a small project that is
%  maintained (it has recent commits), so we can communicate with the authors.}

\section{Comparison to Deflaker}
Deflaker~\cite{deflaker}, like NonDex, reports tests that could be flaky. 
We were unable to run DeFlaker on Randoop, because Deflaker works with
projects built with Maven, but Randoop
uses Gradle for its build system. We plan to convert Randoop's build system to Maven and
perform this case study.
%Ideally, we would have 
%liked to run all three tools (NonDex, DeFlaker, and \TheDeterminismChecker) on an open source project
%that wasn't explored by any of the tool for a fair comparison. Unfortunately, there are issues with the
%DeFlaker plugin and we weren't able to run it on any project. We have created a Github issue for the same
%in its repository.
%\todo{Deflaker works now. Tried on retrofit: only works where there is sufficient change in code coverage between head and its parent.
%Looked throught= about 10 commits, didn't find differential code coverage and therefore no flakes}
%\todo{Deflaker requires Maven build. Randoop is on Gradle.}

\subsection{\TheDeterminismChecker on DeFlaker benchmarks}\label{sec:deflaker-benchmarks}
DeFlaker found 87 previously unknown flaky tests in 93 projects that were being actively developed at the time
the paper was written. The authors reported 19 of these tests, out of which 7 were addressed by the maintainers of
those projects~\cite{deflaker}.
We ran \theDeterminismChecker on the part of each of these
codebases where the reported bug was fixed, as in \cref{sec:nondex-benchmarks}.
% The DeFlaker authors graciously fixed problems we discovered while using
% their tool.

Four of the seven flaky tests (two in \<achilles>, one each in \<jackrabbit-oak> and \<togglz>)
were caused by a race condition, which \theDeterminismChecker
cannot detect.  (This is a strength of DeFlaker over \theDeterminismChecker.)
We were unable to build \<togglz> and \<nutz> which had one flakiness issue each.
However, we extracted the source code causing the flakiness in these repositories into test cases
after looking at the corresponding issues on Github. The bug in \<togglz> was caused by a copy method that iterated over
an \<OrderNonDet Set> and expected it to be deterministic. \TheDeterminismChecker correctly flagged
an error in the loop that iterated over the \<Set>.
The flakiness in \<nutz> was a result of printing a response received over http.
Since network operations are nondeterministic, we annotated the method in \<nutz> that
returns this response as \<NonDet> which lead \TheDeterminismChecker to report an error at the print statement. 
\TheDeterminismChecker was able to find the source of flakiness in \<checkstyle>. This bug was in a test
case that treated an array returned by \<Class.getDeclaredConstructors()>
as deterministic.
