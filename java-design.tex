\section{Determinism types for Java}\label{sec:java-types}

In Java, a type qualifier is written as a type annotation (which has a leading \<@>), as in
\<@Positive int>.

Our implementation
handles array types, and any class (including user-defined classes) that
implements the \<Iterable> or \<Iterator> interfaces; this includes all
Java collections such as \<List>s and \<Set>s.  For simplicity, and because
the semantics and rules are similar even though the syntax varies, this
paper uses the term ``collection'' and the type \<Collection> to represent
all these types.


(In Java,
\<HashMap> cannot be deterministic, but \<TreeMap> and \<LinkedHashMap>
can.\todo{forward reference?})



\Cref{code-determinism} presents some of the JDK methods that we have annotated with our determinism types and give examples of
client code that would produce errors at compile time.
\begin{figure}
    \begin{verbatim}
    // Annotated JDK methods
    public class Random implements java.io.Serializable {
        public @NonDet Random() {}
    }
    public class PrintStream extends FilterOutputStream 
        implements Appendable, Closeable {
        public void println(@Det Object x) {}
    }
    
    // Client code
    class Client {
        void test() {
            @Det double d = Math.random(); // Error - subtyping rules violated.
            @NonDet double nd = Math.random(); // No error.
            System.out.println(nd); // Error - println takes @Det arguments.
        }
    }
    \end{verbatim}
    \caption{Example: Errors detected by \theDeterminismChecker.}
\todo{Use real examples of bugs from case studies, throughout.  Artificial
  examples are less compelling (and lead the reader to wonder whether no
  real example exist).  Put the first example of an error in a figure in
  Section 1, to help with motivating the problem.}
    \label{code-determinism}
\end{figure}


\begin{itemize}
\item
In \theDeterminismCheckerImplementation,
type polymorphism is handled by Java's generics mechanism, which
\theDeterminismChecker fully supports, including class and method generics,
inference, etc.
Given the Java declaration
\begin{Verbatim}
<T> T identity(T arg) { return arg; }
\end{Verbatim}
the type of \<identity(arg)> is the same as the type of
\<arg>, and a programmer can write a type such as \codeid{List<@NonDet
  Integer>} (but see below for details about collections).
\item
Basetype polymorphism is enabled by writing a type qualifier on a use of a
type variable, which overrides the type qualifier at the instantiation
site.
For example, a \<choose> operation on sets could be defined in Java as
\begin{Verbatim}
class ChooseableSet<T> implements Set<T> {
  // return an arbitrary element from the set
  @NonDet T choose();
}
\end{Verbatim}
\item
Java does not provide a syntax that can be used for qualifier polymorphism,
so we use a special type qualifier name, \<@PolyDet>.
A qualifier-polymorphic method \<m> with signature $\forallt{\kappa} \kappa\ \<int> \times \<Det boolean> \rightarrow
\kappa\ \<String>$ would be written in our dialect of Java as
\begin{Verbatim}
@PolyDet String m(@PolyDet int, @Det boolean)
\end{Verbatim}
Each use of \<@PolyDet> stands for a use of the qualifier variable
$\kappa$, and there is no need to declare the qualifier variable $\kappa$.
\TheDeterminismChecker currently supports
qualifier
polymorphism on methods but not on classes.
Therefore, \<@PolyDet> may be written in methods (signatures and bodies)
but not on fields.
\end{itemize}


\subsection{Defaulting rules}\label{defaulting}

\TheDeterminismChecker applies a default qualifier at each unannotated type
use (except uses of type variables, which already stand for a type that was
defaulted at the instantiation site).
This does not change the expressivity of the type system; it merely makes
the system more convenient to use.

In \theDeterminismChecker, we chose \<PolyDet> as the default for all 
method parameters and return types including constructors.
The rationale for this design choice was to not make \theDeterminismChecker too restrictive by making everything \<Det>.  This implies that
\theDeterminismChecker still allows some non-determinism as long as it doesn't break any determinism specifications explicitly written by the programmer.

\begin{verbatim}
@PolyDet int foo(@PolyDet int x) { ... }

void bar(@NonDet int random, @Det int value) {
    @NonDet int x = foo(random);      // No error.
    @Det int y = foo(value);          // No error.
    System.out.println(foo(random));  // Error.
}
\end{verbatim}
In the example above, the method \codeid{foo} is annotated with the \<PolyDet> type qualifier which allows
the caller method \codeid{bar} to invoke it with both \<Det> and \<NonDet> arguments. \TheDeterminismChecker flags an error 
only when the returned value is used in a way that breaks any deterministic specification (Printing a \<NonDet> value
breaks our determinism guarantees).


\<@Det> equality is with respect to \<.equals()>.


\subsection{Polymorphism}\label{java-polymorphism}

\OurTypeSystem supports three types of polymorphism:  type
polymorphism, basetype polymorphism, and qualifier polymorphism.
These apply to both classes and methods.

\begin{itemize}
\item
In \theDeterminismCheckerImplementation,
type polymorphism is handled by Java's generics mechanism, which
\theDeterminismChecker fully supports, including class and method generics,
inference, etc.
Given the Java declaration
\begin{Verbatim}
<T> T identity(T arg) { return arg; }
\end{Verbatim}
the type of \<identity(arg)> is the same as the type of
\<arg>, and a programmer can write a type such as \codeid{List<@NonDet
  Integer>} (but see below for details about collections).
\item
Basetype polymorphism is enabled by writing a type qualifier on a use of a
type variable, which overrides the type qualifier at the instantiation
site.
For example, a \<choose> operation on sets could be defined in Java as
\begin{Verbatim}
class ChooseableSet<T> implements Set<T> {
  // return an arbitrary element from the set
  @NonDet T choose();
}
\end{Verbatim}
\item
Java does not provide a syntax that can be used for qualifier polymorphism,
so we use a special type qualifier name, \<@PolyDet>.
A qualifier-polymorphic method \<m> with signature $\forallt{\kappa} \kappa\ \<int> \times \<Det boolean> \rightarrow
\kappa\ \<String>$ would be written in our dialect of Java as
\begin{Verbatim}
@PolyDet String m(@PolyDet int, @Det boolean)
\end{Verbatim}
Each use of \<@PolyDet> stands for a use of the qualifier variable
$\kappa$, and there is no need to declare the qualifier variable $\kappa$.
\TheDeterminismChecker currently supports
qualifier
polymorphism on methods but not on classes.
Therefore, \<@PolyDet> may be written in methods (signatures and bodies)
but not on fields.
\end{itemize}





When a user writes a polymorphic qualifier on a method signature or a type parameter,
it indicates that it could be instantiated
%\todo{Should all occurrences of ``resolve'' be ``can be instantiated''} 
to any type in the type system depending on how it is used.
In \theDeterminismChecker, we define a polymorphic qualifier \<PolyDet>.
One of the most common locations for a polymorphic qualifier is at a method signature.
Consider the following declaration of addition:
\begin{verbatim}
@PolyDet int plus(@PolyDet int a, @PolyDet int b) { return a+b; }
\end{verbatim}
\todo{It has the standard meaning.  There are two ways to interpret this.}
There are two valid instantiations for this declaration:
\begin{Verbatim}
  @NonDet int plus(@NonDet int a, @NonDet int b) { return a+b; }
  @Det int plus(@Det int a, @Det int b) { return a+b; }
\end{Verbatim}
This indicates that method \<foo> can be called with arguments having any of the \codeid{Det or NonDet} type qualifiers.
%   (\<@OrderNonDet> is not allowed because it is invalid on primitive types).
 \<PolyDet> gets instantiated to the least upper bound of
the actual types on arguments. For instance, if this method is called as \codeid{foo(@Det String arg1, @NonDet boolean arg2)}, \theDeterminismChecker instantiates the method declaration as \codeid{@NonDet int foo(@NonDet String param1, @NonDet boolean param2)}
causing the return type to have the type qualifier \<NonDet>.
