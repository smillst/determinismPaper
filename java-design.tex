\section{Determinism types for Java}\label{sec:java-types}

We have implemented our type system for Java, in a tool named \theDeterminismChecker.
The implementation consists of 1200 lines of Java
code, plus 1800 lines of tests (all line measurements are non-comment,
non-blank lines), a manual, etc.
\TheDeterminismChecker requires Java version 8.
It is publicly available at
\myurl{URL for \theDeterminismChecker}.


\subsection{Determinism type qualifiers}

A type qualifier is written in Java source code as a type annotation.
A type annotation has a leading ``\<@>'' and is written immediately before a
Java basetype, as in \<@Positive int> or \codeid{@NonEmpty List<@NonNull String>}.

\TheDeterminismChecker supports the type qualifiers \<@NonDet>,
\<@OrderNonDet>, and \<@Det> (and others described below).
The meaning of \<@Det> is with respect to value
equality, not reference equality; that is,
values on different executions are the same with respect to \<.equals()>,
not \<==>.

\TheDeterminismChecker treats arrays analogously to collections:
the semantics and rules are similar even though the syntax varies.
For simplicity this
paper uses the term ``collection'' and the type \<Collection> to represent
collections and arrays.
Collections include any type (including user-defined classes) that
implements the \<Iterable> or \<Iterator> interfaces; this includes all
Java collections such as \<List>s and \<Set>s.



\subsection{Java collection types}\label{sec:maps-java}

A \<Map> is deterministic if its \<entrySet> is deterministic.
In other words, iterating over this \<entrySet> produces the same values in the same order across executions.
The determinism qualifier on the return type of \<entrySet()> is the
same as that on the receiver (\cref{code-determinism}).
That is, its type (ignoring type arguments) is
$\forallt{\kappa} \kappa\ \<Map> \rightarrow \kappa\ \<Set>$\ .

We have not yet found any implementation of the \<Map>
interface in which \<entrySet()> \<keySet()>, and \<values()> behave
differently with respect to determinism:  either all three methods, or none, are deterministic.

The most widely used \<Map> implementations have the following properties:
\begin{itemize}
    \item \<HashMap> is implemented in terms of a hash table, which never
      guarantees deterministic iteration over its entries. A \codeid{@Det
        HashMap} does not exist.
    \item \<LinkedHashMap>, like \<List>, can have any of the \aNonDet,
      \aOrderNonDet, or \aDet type qualifiers. Iterating over a
      \<LinkedHashMap> returns
    its entries in the order of their insertion. An \aOrderNonDet\
    \<LinkedHashMap> can be created by passing an
    \aOrderNonDet\ \<HashMap> to its constructor, as in \<new
    LinkedHashMap(myOndHashMap)>.
    \item \<TreeMap> can either be \<@Det> or \<@NonDet>. An \<@OrderNonDet
      TreeMap> doesn't exist because the entries are always sorted.
\end{itemize}

\noindent
\TheDeterminismChecker prohibits the creation of a \codeid{@Det HashMap} or an \codeid{@OrderNonDet TreeMap}.
\Cref{fig-creation-rules} formalizes the type well-formedness rules for
Java \<Map>s and \<Set>s.

\begin{figure}
    $\infer[\rulename{valid hashmap}]{\vdash: \|\kappa\ \<HashMap>|}{\kappa \in \{ \<OrderNonDet>, \<NonDet> \}}$
    
    \bigskip
    
    $\infer[\rulename{valid treemap}]{\vdash: \|\kappa\ \<TreeMap>|}{\kappa \in \{ \<Det>, \<NonDet> \}}$
    
    \bigskip
    
    $\infer[\rulename{valid hashset}]{\vdash: \|\kappa\ \<HashSet>|}{\kappa \in \{ \<OrderNonDet>, \<NonDet> \}}$
    
    \bigskip
    
    $\infer[\rulename{valid treeset}]{\vdash: \|\kappa\ \<TreeSet>|}{\kappa \in \{ \<Det>, \<NonDet> \}}$
    \caption{Type well-formedness rules for subclasses of \<Map> and \<Set>.}
    \label{fig-creation-rules}
\end{figure}


\subsection{Polymorphism}\label{java-polymorphism}

\OurTypeSystem supports three types of polymorphism:  type
polymorphism, basetype polymorphism, and qualifier polymorphism.
These apply to both classes and methods.

\begin{itemize}
\item
In \theDeterminismCheckerImplementation,
type polymorphism is handled by Java's generics mechanism, which
\theDeterminismChecker fully supports, including class and method generics,
inference, etc.
Given the Java declaration
\begin{Verbatim}
<T> T identity(T param) { return param; }
\end{Verbatim}
the type of \<identity> is $\forallt{\tau} \tau \rightarrow \tau$, and
the type of \<identity(x)> is the same as the type of
\<x>.  A programmer can also write a type qualifier on a type argument,
as in \codeid{@NonDet List<@NonDet Integer>}.
\item
Basetype polymorphism is enabled by writing a type qualifier on a use of a
type variable, which overrides the type qualifier at the instantiation
site.
For example, a \<choose> operation on sets could be defined in Java as
\begin{Verbatim}
interface ChooseableSet<T> implements Set<T> {
  // return an arbitrary element from the set
  @NonDet T choose();
}
\end{Verbatim}
\item
Java does not provide a syntax that can be used for qualifier polymorphism,
so \theDeterminismChecker uses a special type qualifier name, \<@PolyDet>.  (\<@PolyDet> stands
for ``polymorphic determinism qualifier''.)
A qualifier-polymorphic method \<m> with signature
$\forallt{\kappa} \kappa\ \<int> \times \<Det boolean>
\rightarrow \kappa\ \<String>$ would be written as
\begin{Verbatim}
@PolyDet String m(@PolyDet int, @Det boolean)
\end{Verbatim}
Each use of \<@PolyDet> stands for a use of the qualifier variable
$\kappa$, and there is no need to declare the qualifier variable $\kappa$.
\TheDeterminismChecker currently supports
qualifier
polymorphism on methods but \emph{not} on classes.
Therefore, \<@PolyDet> may be written in methods (signatures and bodies)
but not on fields.
\end{itemize}

Qualifier  polymorphism is common on methods that a programmer might think
of as deterministic.  For example, an addition method should be defined as

\begin{verbatim}
    @PolyDet int plus(@PolyDet int a, @PolyDet int b) { return a+b; }
\end{verbatim}

\noindent
This can be used in more contexts than

\begin{verbatim}
    @Det int plus(@Det int a, @Det int b) { return a+b; }
\end{verbatim}

\noindent
can be, as was shown in \cref{sec:basic-polymorphism}.


Just as a qualifier variable $\kappa$ is written as \<@PolyDet> in Java
source code, $\up{\kappa}$ is written as \<@PolyDet("up")>, and
$\down{\kappa}$ is written as \<@PolyDet("down")>.  An occurrence of a
qualifier variable that does not affect the binding of that variable
(\cref{bindings-uses}) is written \<@PolyDet("use")>.


All of our syntax is legal Java code that can be compiled with any Java 8
or later compiler.


\Cref{code-determinism} presents some of the JDK methods that we have annotated with our determinism types and give examples of
client code that would produce errors at compile time.

%\begin{figure}
%    \begin{verbatim}
%    // Annotated JDK methods
%    public class Random implements java.io.Serializable {
%        public @NonDet Random() {}
%    }
%    public class PrintStream extends FilterOutputStream 
%        implements Appendable, Closeable {
%        public void println(@Det Object x) {}
%    }
%    
%    // Client code
%    class Client {
%        void test() {
%            @Det double d = Math.random(); // Error - subtyping rules violated.
%            @NonDet double nd = Math.random(); // No error.
%            System.out.println(nd); // Error - println takes @Det arguments.
%        }
%    }
%    \end{verbatim}
%    \caption{Example: Errors detected by \theDeterminismChecker.}
%\todo{Use real examples of bugs from case studies, throughout.  Artificial
%  examples are less compelling (and lead the reader to wonder whether no
%  real example exist).  Put the first example of an error in a figure in
%  Section 1, to help with motivating the problem.}
%    \label{code-determinism}
%\end{figure}

\begin{figure}
    \begin{verbatim}
    // Annotated JDK methods
    public interface Map<K,V> {
        @PolyDet Set<Map.Entry<K, V>> entrySet(@PolyDet Map<K,V> this);
    }
    public interface Iterator<E> {
        @PolyDet("up") E next(@PolyDet Iterator<E> this);
    }
    
    // Client code
    public class MapUtils {
        public static <K extends @PolyDet Object, V extends @PolyDet Object> 
                                @Det String toString(@PolyDet Map<K,V> map) {
            ...
            for (Map. @Det Entry<K,V> entry : map.entrySet()) { ... }
        }
    }
    [ERROR] MapUtils.java:[20,50] [enhancedfor.type.incompatible] 
    incompatible types in enhanced for loop.
    found : @PolyDet Entry<K extends @PolyDet Object,V extends @PolyDet Object>
    required: @Det Entry<K extends @PolyDet Object,V extends @PolyDet Object>
    \end{verbatim}
    \caption{Example: Error detected by \theDeterminismChecker in \<scribe-java>~\cite{nondex}.}
%    \todo{Use real examples of bugs from case studies, throughout.  Artificial
%        examples are less compelling (and lead the reader to wonder whether no
%        real example exist).  Put the first example of an error in a figure in
%        Section 1, to help with motivating the problem.}
    \label{code-determinism}
\end{figure}


\subsection{Defaulting}\label{defaulting}

\TheDeterminismChecker applies a default qualifier at each unqualified Java
basetype (except uses of type parameters, which already stand for a type that was
defaulted at the instantiation site where a type argument was supplied).
This does not change the expressivity of the type system; it merely makes
the system more convenient to use by reducing programmer effort and code clutter.
Defaulted type qualifiers are not trusted:  they are type-checked just as
explicitly-written ones are.

Formal parameter and return types default to \<@PolyDet>.  That is, a
programmer-written method

\begin{Verbatim}
  int plus(int a, int b) { ... }
\end{Verbatim}

\noindent
is treated as if the programmer had written

\begin{Verbatim}
  @PolyDet int plus(@PolyDet int a, @PolyDet int b) { ... }
\end{Verbatim}

\noindent
and its function type is

$\forallt{\kappa}  \kappa\ \<int> \times \kappa\ \<int> \rightarrow \kappa\ \<int>$

\noindent
This choice type-checks if the method body does not make calls to any
interfaces that require \<@Det> arguments or produce \<@NonDet> results.
Otherwise, the programmer must write explicit type qualifiers in the method
signature.

If a formal parameter or return type is a collection, its element type
defaults to \<@PolyDet>.  No other choice would result in a valid type
after instantiation.

As an exception to the above rules about return types, if a method's formal
parameters (including the receiver) are all \<@Det> --- that is, there are
no unannotated or \<@PolyDet> formal parameters --- then an unannotated
return type defaults to \<@Det>.  This is particularly useful for methods
that take no formal parameters.  A type like $\forallt{\kappa} ()
\rightarrow \kappa\ \<int>$ does not make sense, because there is no basis
on which to choose an
instantiation for the type argument $\kappa$.  Treating the type as $()
\rightarrow \<Det>\ \<int>$ permits just as many uses.

Types are inferred for unannotated local variables; see \cref{sec:dataflow-java}.

The default annotation for other unannotated types is \<@Det>, because
programmers generally expect their programs to behave the same when re-run
on the same inputs.



\subsection{Type refinement via dataflow analysis}\label{sec:dataflow-java}

\todo{
\OurTypeSystem performs local type inference within method bodies.
It does not perform whole-program type inference.
This makes separate compilation possible.
It forces programmers to write specifications (type qualifiers) on methods,
which is good style and valuable documentation.

The local type inference is flow-sensitive.  That is, each variable may
have a different type on every line of the program, depending on
assignments to the variable.  The inferred type always respects any
declared type:  that is, it is always a refinement (a subtype) of the
declared type, if any.
}




\OurTypeSystem is
flow-sensitive~\cite{Hunt:2006:FST:1111037.1111045,Adams:2011:FTR:2048066.2048105,Sui:2016:OSU:2950290.2950296}.
That is, an expression may have
a different type on every line of the program, based on assignments and
side effects.  The current type must always be consistent with the declared
type, if any.
Consider the example below:

\begin{verbatim}
  @NonDet int x = ...;
                          // The type of x is @NonDet int
  x = 42;
                          // The type of x is @Det int
  @Det int y = x;
  x = random();
                          // The type of x is @NonDet int
  y = random();           // Error: y is declared as @Det int
\end{verbatim}

Arbitrary expressions can be given more specific types, including fields
and pure method calls.  A type refinement is lost whenever a side effect
might affect the value.  For example, type refinements to all fields are
lost whenever a non-pure method is called.

This flow-sensitive type refinement achieves local variable inference,
freeing programmers from writing many local variable types.


\subsection{The environment}\label{sec:environment-java}

The return type of \<System.getenv>, which reads an operating system
environment variable, is always \<@NonDet>.

The return type of \codeid{System.getProperty} is usually \<@NonDet>. However,
the result is \<@Det> if the argument is \codeid{"line.separator"}; although
it varies by operating system, many programs including \<diff> and editors hide those
differences from the user.
The result is also deterministic for \codeid{"path.separator"} and
\codeid{"file.separator"}, because these lead to the same behavior given
corresponding environments.
A user of \theDeterminismChecker can specify Java properties that must be passed on the \<java>
command line and thus act like inputs to the program;
\theDeterminismChecker treats these as deterministic.

The inputs to a program are also treated as deterministic.  That is, the type of
the formal parameter to \<main> is
\<@Det String @Det []>, a deterministic array of deterministic strings.


% LocalWords:  basetype NonEmpty NonDet OrderNonDet Det Iterable entrySet
% LocalWords:  keySet LinkedHashMap myOndHashMap formedness PolyDet getenv
% LocalWords:  expressivity getProperty
