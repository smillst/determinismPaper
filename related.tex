\section{Related work}\label{sec:related}

\todo{TODO}

The state of the art is NonDex~\cite{nondex}.  It is a highly effective tool.
% https://github.com/TestingResearchIllinois/NonDex/wiki/Supported-APIs
NonDex uses a hand-crafted list of 47 methods (25 unique method names)
in 13 classes that were identified as potential sources of flakiness. 
This is analogous to our annotations on the JDK\@. So far, we have annotated
928 methods across 41 classes.
For each of the identified methods, the authors built models that
return different results when called consecutively. A modified JVM then
runs a given test multiple times and reports the test as being flaky if it observes
diverging test output. While this approach produces precise results, it requires manual inspection
and considerable debugging effort to locate and fix the source of flakiness. \TheDeterminismChecker, in
contrast, reports the cause of nondeterminism at compile requiring little to no debugging effort with the
caveat of producing higher false positives compared to NonDex.
%\todo{Check whether our JDK is missing anything they support - No}
%\todo{Did the NonDex authors contact the maintainers of the tools in which
%  they found nondeterminism - No!}
%\todo{The NonDex authors didn't contact the maintainers of tools.
%From the paper: "NonDex detected 57 previously unknown
%flaky tests in open-source projects and three flaky tests
%that have been already fixed by the open-source software
%developers."}



\todo{A NonDet annotation:
  \url{http://www.swi-prolog.org/pldoc/man?section=modes}}

%% Added this info in the experiments section
%\todo{All nondeterminism that NonDex found was due to 7 methods in 3 classes:
%Class\#getDeclaredFields
%Class\#getDeclaredMethods
%Class\#getFields
%DateFormatSymbols\#getZoneStrings
%HashMap\#entrySet
%HashMap\#keySet
%HashMap\#values
%}
