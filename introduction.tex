\section{Introduction}

A nondeterministic program may produce different output on different runs
when provided with the same input.
Nondeterminism is a serious problem for software developers and users.
\todo{Add lots of citations in the below.}
\begin{itemize}
\item
  Nondeterminism makes a program difficult to \textbf{test}, because test
  oracles must account for all possible behaviors while still enforcing
  correct behaviors.  Test oracles that are too strict lead to flaky
% ,Gyori:2015:RTD:2771783.2771793
  tests~\cite{LuoHEM2014,nondex,deflaker,Rahman:2018:IFF:3236024.3275529}
  that sometimes pass and sometimes fail.  Flaky tests must be re-run, or
  developers ignore them; in either case, their utility to detect defects
  is limited.
\item
  Nondeterminism makes it difficult to \textbf{compare} two runs of a
  program on different data, or to compare a run of a slightly modified
  program to an original program.  This hinders debugging and maintenance,
  and prevents use of techniques such as Delta Debugging~\cite{Zeller1999}.
\item
  Nondeterminism may make outputs incompatible with previous outputs relied
  on by users or by other systems~\todo{These do not seem to be supporting
    the claim about bugs related to incompatibilities between outputs.
    These seem to be general citations about test flakiness or dependence.
    Readers will be unhappy if we make a claim but then do not support it.}\cite{Herzig:2015:EDF:2819009.2819018,Huo:2014:IOQ:2635868.2635917,ZhangJWMLEN2014,BellKMD2015,Fowler,Gyori:2015:RTD:2771783.2771793,Dan:2013:10.1007/978-3-642-39038-8_25,Vahabzadeh:2015:7332456,Sudarshan}.
%  \todo{Citations about bugs related to this?}
\item
  Nondeterminism reduces users' and developers' \textbf{trust} in a program's output.
\end{itemize}

Two well-known sources of nondeterminism are concurrency
% (due to OS scheduling or message delivery over a network)
and coin-flipping
(calls to a \<random> API\@).
It may be surprising that nondeterminism is common even in sequential
programs that do not flip coins.
For example, a program that iterates over a hash table
may produce different output on different runs.
So may any program that uses default formatting, such as Java's
\<Object.toString()> which includes a memory address.
Other nondeterministic APIs include date-and-time functions and
accessing system properties such as the file system or environment variables.

We have created an analysis that detects nondeterminism or verifies its
absence in sequential programs.
Our analysis permits a programmer to specify which parts of their program
are intentionally nondeterministic, and it verifies that the remainder is deterministic.
%
If our analysis issues no warnings, then the program produces the same
output when executed twice on the same inputs.  This guarantee is modulo
the limitations of the
analysis, notably concurrency, implicit control flow, and unanalyzed libraries (see \cref{sec:threats}).
%
Our analysis works at compile time, giving a guarantee over every possible
execution of the program, by contrast to unsound dynamic tools that attempt
to discover when a program has exhibited nondeterministic behavior on
specific runs.  There is no need for a special JVM nor rerunning a program
multiple times.
%
Our analysis handles collections that will contain the same values, but
possibly in a different order, on different runs.
%
Our analysis is precise enough for practical use.  It permits calls to
nondeterministic APIs, and only issues a warning if they are used in ways
that may lead to nondeterministic output observed by a user.  Like any
sound analysis, it can issue false positive warnings.



The high-level goal of our work is to provide programmers with a tool for
specifying deterministic properties in a program and verifying them
statically.
%
Other researchers have also recognized the importance of the problem of nondeterminism.
Previous work in program analysis for nondeterminism has focused on unsound dynamic
approaches that identify flaky test cases.
NonDex~\cite{nondex} uses a modified JVM that returns different results on different
executions, for a few key JDK methods with loose specifications.  Running a
test suite multiple times can reveal unwarranted dependence on those
methods.
%NonDex~\cite{nondex} is a tool that manually\todo{This is confusing:  the
%  tool can't manually identify.  Do you mean that the developers manually
%  identified, then the tool uses those models?}
%identifies 
DeFlaker~\cite{deflaker} looks at a range of commit versions
of a code, and marks a test as flaky if it doesn't execute any modified code but still fails in the newer version. These techniques
have been able to identify issues in real-world programs, some of which
have been fixed by the developers. We believe that identifying and
resolving nondeterminism
at an earlier phase in the software development lifecycle would be greatly beneficial to
developers.

Our analysis uses three main abstractions, or approximations to run-time values:
\begin{description}
\item[\<NonDet>] represents values that might differ from run to run.
\item[\<OrderNonDet>] represents collections that are guaranteed to contain
  the same elements, albeit in possibly different iteration order.
\item[\<Det>] represents values that will be the same across executions.
\end{description}
\noindent
Programmers can write these to specify their program's behavior.
Our full analysis contains other features that increase
expressiveness.

\begin{figure}

\noindent
In \<TypeVariable.java>:

\begin{Verbatim}
160:   public List<TypeVariable> getTypeParameters() {
161:-    Set<TypeVariable> parameters = new HashSet<>(super.getTypeParameters());
161:+    Set<TypeVariable> parameters = new LinkedHashSet<>(super.getTypeParameters());
162:     parameters.add(this);
163:     return new ArrayList<>(parameters);
164:   }
\end{Verbatim}

\caption{Fixes made by the Randoop developers in response to our bug report
  about improper use of a HashSet.  Lines starting with ``\<->'' were
 removed and those starting with ``\<+>'' were added.
 Our tool, \theDeterminismChecker, confirmed that 
25 other uses of \<new HashSet> were acceptable, as were 15 uses of \<new HashMap>.}
\label{fig:randoop-bug-hashset}
\end{figure}


This paper makes the following contributions:
\begin{enumerate}
  \item We designed a type system, named \ourTypeSystem, for expressing determinism properties.

  \item We implemented the analysis, as a pluggable type system for Java, in a
    tool called \theDeterminismChecker.

  \item
  \todo{It's a bit weird that we call this a contribution but it is never
    explicitly discussed in the paper.  I think it's probably OK.}
  In a case study, we annotated several libraries, including the collection
    classes and other parts of the JDK (Java's standard library).  This
    provides a formal, machine-readable specification for the libraries, and
    it demonstrates the expressiveness of our type system.

  \item In another case study, we ran our analysis on a 24 KLOC project that
    developers had already spent weeks of testing and inspection effort to
    make deterministic.  \TheDeterminismChecker
    discovered 5 instances of nondeterminism that the developers had
    overlooked.
    The developers fixed all these issues when we reported them.
    \Cref{fig:randoop-bug-hashset,fig:randoop-bug-getenv} show two examples.

  \item We compared our tool, \theDeterminismChecker, against state-of-the-art flaky test
    detectors, on their benchmarks.
    \TheDeterminismChecker found all the non-concurrency nondeterminism
    found by the other tools.

\end{enumerate}

\todo{1211 LOC of Java in the determinism checker}

%\todo{It is essential that the introduction includes an example real-world
%  defect that \theDeterminismChecker found.}


% LocalWords:  NonDex DeFlaker Det OrderNonDet NonDet
