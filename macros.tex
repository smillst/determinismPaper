%%% Todo comments

%% Comment out one of these two definitions.
% \newcommand{\todo}[1]{\relax}
\newcommand{\todo}[1]{{\color{red}\bfseries [[#1]]}}


%%% Anonymization.

\newif\ifanonymous
\anonymoustrue

% For anonymity.  Let's find better names.
\ifanonymous
\newcommand{\ourTypeSystem}{Tendon\xspace}
\newcommand{\OurTypeSystem}{Tendon\xspace}
\newcommand{\theDeterminismChecker}{Gar\c{c}on\xspace}
\newcommand{\TheDeterminismChecker}{Gar\c{c}on\xspace}
\else
\newcommand{\ourTypeSystem}{our type system\xspace}
\newcommand{\OurTypeSystem}{Our type system\xspace}
\newcommand{\theDeterminismChecker}{the Determinism Checker\xspace}
\newcommand{\TheDeterminismChecker}{The Determinism Checker\xspace}
\fi
\newcommand{\theDeterminismCheckerImplementation}{\theDeterminismChecker implementation\xspace}
\newcommand{\TheDeterminismCheckerImplementation}{\TheDeterminismChecker implementation\xspace}

% A commit hash
\newcommand{\commit}[1]{\ifanonymous{ (commit #1)}\else\fi}
\newcommand{\bugOne}{\textbf{Bug 1}}
\newcommand{\bugTwo}{\textbf{Bug 2}}
\newcommand{\bugThree}{\textbf{Bug 3}}
\newcommand{\bugFour}{\textbf{Bug 4}}

%%% Types and type rules

\newcommand{\Det}{\text{\<Det>}\xspace}
\newcommand{\OrderNonDet}{\text{\<OrderNonDet>}\xspace}
\newcommand{\NonDet}{\text{\<NonDet>}\xspace}
\newcommand{\Ond}{\OrderNonDet}

\newcommand{\up}[1]{\ifmmode #1\mathord{\uparrow} \else #1$\uparrow$\fi}
\newcommand{\down}[1]{\ifmmode #1\mathord{\downarrow} \else #1$\downarrow$\fi}

\newcommand{\PolyDetUp}{\<PolyDet$\uparrow$>}
\newcommand{\PolyDetDown}{\<PolyDet$\downarrow$>}

\newcommand{\rulename}[1]{\textsc{#1}}

\newcommand{\wellformed}[1]{\vdash : #1}

% Polymorphic type
\newcommand{\forallt}[1]{\ensuremath{\forall #1 . \ }}


%%% Common types
\newcommand{\CollectionE}{\mbox{\<Collection>\angles{$\tau_e$}}}
\newcommand{\CollectionKB}{\mbox{\<Collection>\angles{$\kappa_e\ \beta_e$}}}
\newcommand{\ListE}{\mbox{\<List>\angles{$\tau_e$}}}
\newcommand{\ListKB}{\mbox{\<List>\angles{$\kappa_e\ \beta_e$}}}


%%% Code formatting

% \|name| or \mathid{name} denotes identifiers and slots in formulas
\def\|#1|{\mathid{#1}}
\newcommand{\mathid}[1]{\ensuremath{\mathit{#1}}}
% \<name> or \codeid{name} denotes computer code identifiers
\def\<#1>{\codeid{#1}}
\protected\def\codeid#1{\ifmmode{\mbox{\sf{#1}}}\else{\sf #1}\fi}
\protected\def\codeid#1{\ifmmode{\mbox{\ttfamily{#1}}}\else{\ttfamily #1}\fi}
\protected\def\codeid#1{\ifmmode{\mbox{\smaller\ttfamily{#1}}}\else{\smaller\ttfamily #1}\fi}

% Enclosed in angle brackets
\newcommand{\angles}[1]{\ensuremath{\langle}#1\ensuremath{\rangle}}
% A parameterized type
\newcommand{\ptype}[2]{\<#1>\ensuremath{\langle}\<#2>\ensuremath{\rangle}}
% A larger (in font size) parameterized type
\newcommand{\lptype}[2]{{\larger\<#1>\ensuremath{\langle}\<#2>\ensuremath{\rangle}}}

\newcommand{\myurl}[1]{\ifanonymous (URL elided)\else\url{#1}\fi}

%%% Citations

\defcitealias{randoop-tool}{Randoop}
